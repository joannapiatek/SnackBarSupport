\section{Podsumowanie}
Realizacja systemu wykorzystującego rozproszoną bazę \textit{MongoDB} oraz mechanizmu replikacji opartego na zestawie replik oferowanego przez tą platformę zakończyło się powodzeniem. Wbudowane w \textit{MongoDB} narzędzia jak \textit{mongo shell} są wystarczające do konfiguracji średnio-zaawansowanych systemów, eliminując koniecznośc wykorzysywania komercyjnych narzędzi służących wdrażaniu \textit{MongoDB}. \\
Mechanizm replikacji okazał się działać zgodnie z oczekiwaniami, co pozwala tworzyć rozproszone systemy, a zatem uzyskać bezpieczeństwo danych oraz wysoką dostępność. \\
Z punktu wykorzystania \textit{MongoDB} ważną kwestią jest aktualność oraz dokumentacji sterowników (ang. \textit{drivers}) dla poszczególnych platform jak .NET czy Node. W tej kwestii nie napotkano na żadne problemy co umożliwiło utworzenia aplikacji klienckiej na platformie ASP.NET.